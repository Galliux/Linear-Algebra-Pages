\documentclass[a4paper]{article}

\title{Ma Spec 100p -- Linjär Algebra -- Kapitel 4.2}
\author{Samuel Bayley Eriksson}
%\institute{SIG}
%\date{\today}

%\usepackage{DefaultPreamble}
%\usepackage[orientation=landscape,size=a0,scale=1.4,debug,grid]{beamerposter}
\usepackage{amsfonts, amsmath, amssymb, bm, cellspace,  enumitem, extarrows, graphicx, mathtools, multicol, multirow, parskip, systeme, tabularx, tcolorbox, tikz, tikz-3dplot, varwidth, xfrac}
\usepackage[margin=1in]{geometry}
\usepackage{tasks}
\sysdelim.. % Defines left and right delimiter for \systeme{}. Change "." to \{ or \[ for example

% LaTeX Configuration
\arraycolsep=3.75pt
\let\labelenumi\theenumi
\renewcommand{\labelenumi}{\alph{enumi})}
% Defined Commands
\newlength{\arrow}
\newcommand*{\loongleftarrow}[1]{
	\begingroup
	\settowidth{\arrow}{\scriptsize$#1$}
	\xleftarrow{\mathmakebox[\arrow]{}}
	\endgroup
}
%\newcommand<>{\alertbold}[1]{\textcolor{red}{\textbf{#1}}}
%\newcommand<>{\Sfrac}[2]{\text{\sfrac{#1}{#2}}}
%\newcommand<>{\bvec}[1]{\normalfont\textbf{#1}}
\newcommand{\bvec}[1]{\normalfont\textbf{#1}}
%\newcommand<>{\span}[1]{\text{Span\{$#1$\}}}
%\newcommand<>{\col}[1]{\mathrm{Col\,}#1}
%\newcommand<>{\nul}[1]{\mathrm{Nul\,}#1}
%\newcommand<>{\rank}[1]{\mathrm{rank\,}#1}
%\newcommand<>{\blfootnote}[1]{%
%  \begingroup
%  \renewcommand\thefootnote{}\footnote{#1}%
%  \addtocounter{footnote}{-1}%
%  \endgroup
%}

% Task Configuration
\settasks{label-offset=1em}

%% Beamer Configuration
%\usefonttheme{serif}
%% Available fonts: structurebold, structurebolditalic, structuresmallcapsserif, structureitalicsserif, serif, default
%\usetheme{Madrid}
%\usecolortheme{orchid}
%\setbeamertemplate{navigation symbols}{} % Hides navigation symbols
%\setbeamertemplate{footline}{} % Hides the bottom row of information such as title, institution etc.
%\setitemize{label=\usebeamertemplate{itemize item}} % Sets the default itemization bullet to beamer default circle. Does not seem to display any bullet if this is not here.

% TColorBox Configuration
%\tcbuselibrary{skins}
\tcbuselibrary{most}
\newtcolorbox{subblock}[2][]{
code={
\usebeamercolor{block title}
\colorlet{titlebg}{block title.bg}
\colorlet{titlefg}{block title.fg}
\usebeamercolor{block body}
\colorlet{bodybg}{block body.bg}
\colorlet{bodyfg}{block body.fg}
},
coltitle=block title.fg, % Text color in title
colbacktitle=block title.bg, % Background color of title
colupper=block body.fg, % Text color in body
colback=block title.bg!10, % Background color of body
%colframe=red, % Border color
lower separated=false, % Removes dashed line between upper and lower
frame empty, % Hides border of body?
title=#2, % Sets the title to argument 2
boxrule=0pt,
rounded corners, % Boolean : Rounded corners
fonttitle=\large, % Sets font-size for title
before=\par\medskip\noindent, % Creates spacing before block is started
%after=\par\medskip\noindent, % Spacing after block is finished
top=2pt,
left=0pt, % Sets left indentation
%math, % Enters math-mode in upper and lower
%math lower, % Enters math-mode in lower
%ams gather*, % Enters gather*-mode from amsmath immediately in body
%box align=center,
%halign title=left, % Alignment of title
%halign upper=left, % Alignment of upper
show bounding box, % Displays bounding box
#1 % Code for body
}

% TikZ Configuration
\usetikzlibrary{angles, arrows.meta, calc, tikzmark}
\newdimen\numht
\newdimen\numwd
\settoheight{\numht}{(}
\settowidth{\numwd}{0}
%
\begin{document}
\maketitle
I uppgift 7--14 använd antingen ett lämpligt teorem för att visa att den givna mängden $W$ är ett vektorrum, eller hitta ett exempel som motbevisar det.
\setcounter{task}{6}
\begin{tasks}[resume, label={\bf{\arabic*.}}](2)
\task $\left\lbrace\left[\begin{array}{c}a \\ b \\ c\end{array}\right]\colon a+b+c=2\right\rbrace$
\task $\left\lbrace\left[\begin{array}{c}r \\ s \\ t\end{array}\right]\colon 5r-1=s+2t\right\rbrace$
\task $\left\lbrace\left[\begin{array}{c}a \\ b \\ c \\ d\end{array}\right]\colon\begin{aligned}&a-2b=4c\\&2a=c+3d\end{aligned}\right\rbrace$
\task $\left\lbrace\left[\begin{array}{c}a \\ b \\c \\ d\end{array}\right]\colon\begin{aligned}&a+3b=c\\&b+c+a=d\end{aligned}\right\rbrace$
\task $\left\lbrace\left[\begin{array}{c}b-2d \\ 5+d \\ b+3d \\ d\end{array}\right]\colon b,d\in\mathbb{R}\right\rbrace$
\task $\left\lbrace\left[\begin{array}{c}b-5d \\ 2b \\ 2d+1 \\ d\end{array}\right]\colon b,d\in\mathbb{R}\right\rbrace$
\task $\left\lbrace\left[\begin{array}{c}c-6d \\ d \\ c\end{array}\right]\colon c,d\in\mathbb{R}\right\rbrace$
\task $\left\lbrace\left[\begin{array}{c}\systeme{-a+2b, a-2b, 3a-6b}\end{array}\right]\colon a,b\in\mathbb{R}\right\rbrace$
\end{tasks}
\medskip
\setcounter{task}{30}
\begin{tasks}[resume, label={\bf{\arabic*.}}](1)
\task Definiera $T\colon\mathbb{P}_2\to\mathbb{R}^2$ som $T(\bvec{p})=\left[\begin{array}{c}\bvec{p}(0) \\ \bvec{p}(1)\end{array}\right]$.
\begin{enumerate}[label={\alph*)}]
\item Visa att $T$ är en linjär transformation. [\emph{Hint:} För godtyckliga polynom $\bvec{p},\bvec{q}$ i $\mathbb{P}_2$, beräkna $T(\bvec{p}+\bvec{q})$ och $T(\bvec{cp})$.]
\item Hitta ett polynom $\bvec{p}$ i $\mathbb{P}_2$ som spänner upp kärnan till $T$ och beskriv värdemängden till $T$.
\end{enumerate}
\task Definiera en linjär transformation $T\colon\mathbb{P}_2\to\mathbb{R}^2$ som $T(\bvec{p})=\left[\begin{array}{c}\bvec{p}(0) \\ \bvec{p}(0)\end{array}\right]$. Hitta polynom $\bvec{p}_1$ och $\bvec{p}_2$ i $\mathbb{P}_2$ som spänner upp kärnan till $T$ och beskriv värdemängden till $T$.
\task Låt $M_{2\times2}$ vara vektorrummet av alla $2\times2$ matriser och definiera $T\colon M_{2\times2}\to M_{2\times2}$ som $T(A)=A+A^T$.
\begin{enumerate}[label={\alph*)}]
\item Visa att $T$ är en linjär transformation.
\item Låt $B$ vara godtyckligt element ur $M_{2\times2}$ så att $B^T=B$. Hitta ett $A$ i $M_{2\times2}$ så att $T(A)=B$.
\item Visa att värdemängden till $T$ är mängden av alla $B$ i $M_{2\times2}$ där $B^T=B$.
\item Beskriv kärnan till $T$.
\end{enumerate}
\task Låt $V$ och $W$ vara vektorrum och låt $T\colon V\to W$ vara en linjär transformation.  Givet ett delrum $U$ till $V$, låt $T(U)$ beteckna mängden av alla bilder på formen $T(\bvec{x})$ där $\bvec{x}\in U$. Visa att $T(U)$ är ett delrum till $W$.
\setcounter{task}{35}
\end{tasks}
\end{document}